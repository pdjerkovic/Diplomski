\documentclass{beamer}
%\usepackage[demo]{graphicx}
%\usepackage{subfig}
%\usepackage{caption}
\usepackage[utf8]{inputenc}
\usepackage{amssymb}
\usepackage[english,serbian]{babel}
\usetheme{Warsaw}
\setbeamertemplate{caption}{\raggedright\insertcaption\par}
\usepackage{multirow}

%\usepackage{subcaption}
%\usetheme{fibeamer}
\title{Konvolucione mreže super rezolucije}
\subtitle{Specijalistički rad}
\author{Petar Đerković}
\institute{Prirodno-matematički fakultet, Podgorica}
\date{\today}

\begin{document}

\begin{frame}
\titlepage
\end{frame}

%\section{Section 1}
%\subsection{sub a}

\begin{frame}
\frametitle{Osnovni koncepti}
  
\begin{itemize}
\item Problem super rezolucije (SR)
\begin{itemize}
\item jednoulazni \checkmark
\item multiulazni
\end{itemize}
\item Tehnike rijetkog kodiranja
\item Kanal SR obrade
\begin{itemize}
\item Dijeljenje slike na regione (segmente)
\item Pretvaranje segmenata u rječnike male rezolucije (LR)
\item Prelaz u rječnike visoke rezolucije (HR) i rekonstrukcija
\end{itemize}
\item Mana: segregacija komponenti - nepostojanje zajedničke optimizovane strukture
\end{itemize}
\end{frame}


\begin{frame}
\frametitle{Algoritmi SR-a}
  
\begin{itemize}
\item Prediktivni modeli
\item Ivični modeli
\item Statistički modeli
\item Modeli uzorkovanja - najbolji!
\begin{itemize}
\item Unutrašnji modeli uzorkovanja
\item Spoljašnji modeli uzorkovanja
\end{itemize}
\item Danas postoje mnogobrojni pristupi
\item Jednokanalni sistemi boje
\item Izdvojene ocjene: PSNR i SSIM
\end{itemize}
\end{frame}


\begin{frame}
\frametitle{Konvolucione mreže}
  
\begin{itemize}
\item Veoma bitne na polju mašinske vizije
\item Struktura vidnog korteksa
\item Konvolucioni sloj i dijeljenje težina
\item Posebno 3D raspoređivanje neurona i receptivno polje RP
\item Mape bitnih odlika (engl. feature maps)
\item Hiperparametri i ReLU funkcija $f(x)=x^{+}=\max(x,0)$
\end{itemize}


\begin{figure}[ht]
        \begin{minipage}[b]{0.5\linewidth}
            \centering
            \includegraphics[width=1\textwidth]{SLIKE/konvolucija}
            \caption{Primjer konvolucije}
            \label{fig:a}
        \end{minipage}
        \hspace{0.5cm}
        \begin{minipage}[b]{0.4\linewidth}
            \centering
            \includegraphics[width=1\textwidth]{SLIKE/receptive}
            \caption{RP - crveno}
            \label{fig:b}
        \end{minipage}
    \end{figure}

\end{frame}


\begin{frame}
\frametitle{Konvolucione mreže super rezolucije}
\begin{itemize}
\item Benchmark metod (Dong et. al)
\item Mapiranje s kraja na kraj, između LR i HR slike
\begin{itemize}
\item Preprocesiranje bikubičnom intrepolacijom (BI)
\item Izvlačenje i predstavljanje segmenata
\begin{block}{Y - preprocesirani ulaz, $W_1$ - filteri 1. sloja, $B_1$ - bias}
%\begin{equation}
 $ F_1(Y) = \max(0, W_1 \star Y + B_1)$
%\end{equation}
\end{block}
\item Nelinearno mapiranje
\begin{block}{$W_2$ - filteri 2. sloja, $B_2$ - bias}
%\begin{equation}
 $ F_2(Y) = \max(0, W_2 \star F_1(Y) + B_2)$
%\end{equation}
\end{block}
\item Rekonstrukcija
\begin{block}{$W_3$ - filteri 3. sloja, $B_3$ - bias}
%\begin{equation}
 $F_3(Y) =  W_3 \star F_3(Y) + B_3$
%\end{equation}
\end{block}

\end{itemize}
\end{itemize}
\end{frame}


\begin{frame}
\frametitle{Konvolucione mreže super rezolucije}
\begin{itemize}
%\item Postavke hiperparametara
\begin{block}{Postavke hiperparametara}
%\begin{equation}
 $f_1 = 9, f_2 = 1, f_3 = 5, n_1 = 64 \text{ i } n_2 = 32$
%\end{equation}
\end{block}
\item Bez dopune (padding-a) kako izlaz ne bi imao granice
\item Jednostavnija struktura, brža mreža
\end{itemize}


\begin{figure}[h]
\includegraphics[width=\textwidth]{SLIKE/figure2}
\centering
\caption{Vizuelna reprezentacija SRCNN mreže}
\label{fig:srcnn}
\end{figure}

\end{frame}

\begin{frame}
\frametitle{Analogija sa SC}
\begin{itemize}
\item Izvučeni $f_1 \times f_1$ segmenti iz početne slike
\item Algoritam SC daje nam rječnik male rezolucije veličine $n_1$
\item Nađi $n_2$ rijetki skup koeficijenata LR rječnika - to je HR rječnik
\item Rekonstrukcija HR segmenata od HR rječnika i postprocesiranje
\end{itemize}
\begin{figure}[h]
\includegraphics[width=\textwidth]{SLIKE/figure3}
\centering
\caption{SC tehnike iz ugla CNN mreže}
\label{fig:rep}
\end{figure}

\end{frame}



\begin{frame}
\frametitle{Treniranje}
\begin{itemize}
\item Učenje $\Theta = \{W_1 , W_2 , W_3 , B_1 , B_2 , B_3\}$
\item Na skupu Basic: 91 slika $\rightarrow$  24800 subslika $X_i$
\item LR ulaz $Y_i$: zamućenje Gausovim filterom i BI up i down semplovanje $X_i$
\item Funkcija gubitka MSE minimalizovana stohastičnim gradijentnim spustom
\begin{block}{Ažuriranje težina; $l \in \{ 1, 2, 3 \}$ je indeks sloja, a $i$ broj iteracije. $\eta$ je stopa učenja i $n$ broj uzoraka}
%\begin{equation}
 $L(\Theta) = \frac{1}{n} \sum\limits_{i=1}^{n}{||F(Y_i; \Theta) - X_i ||}^2;$
 $\Delta_{i+1} = 0.9 \cdot \Delta_i - \eta \cdot \frac{\partial L}{\partial W_{i}^{l}}, \ \ W_{i+1}^{l}=W_{i}^{l}+\Delta_{i+1}$
%\end{equation}
\end{block}
\item Stopa učenja $10^{-4}$ za prva dva sloja, $10^{-5}$ za zadnji; težine inicijalizovane sa  $\mathcal{N}(0, 0.001)$, a bias na $0$
\end{itemize}
\end{frame}


 \begin{frame}
\frametitle{Rezultati ispitivanja}

 \begin{figure}[ht]
        \begin{minipage}[b]{0.45\linewidth}
            \centering
            \includegraphics[width=1\textwidth]{SLIKE/figure4}
            \caption{Treniranje na većim bazama popravlja performanse}
            \label{fig:a}
        \end{minipage}
        \hspace{0.5cm}
        \begin{minipage}[b]{0.45\linewidth}
            \centering
            \includegraphics[width=1\textwidth]{SLIKE/figure7}
            \caption{Razumno veći filteri 2. sloja su bolji}
            \label{fig:b}
        \end{minipage}
    \end{figure}

\begin{table}[h!]
\centering
\begin{tabular}{ |c|c|c|c|c|c| } 
\hline
\multicolumn{2}{|c|}{$n_1=128$} & \multicolumn{2}{|c|}{$n_1=64$} & \multicolumn{2}{|c|}{$n_1=32$} \\
\multicolumn{2}{|c|}{$n_2=64$} & \multicolumn{2}{|c|}{$n_2=32$} & \multicolumn{2}{|c|}{$n_1=16$} \\
\hline
PSNR & Vrijeme(sec) & PSNR & Vrijeme(sec) & PSNR & Vrijeme(sec) \\
\hline
32.60 & 0.60 & 32.52 & 0.18 & 32.26 & 0.05 \\
\hline
\end{tabular}
\caption{Rezultati treniranja mreže sa različitim brojem filtera - trade-off između brzine i kvaliteta}
\label{table:1}
\end{table}


\end{frame}

\begin{frame}
\frametitle{Rezultati ispitivanja}
\begin{figure}[ht]
        \begin{minipage}[b]{0.45\linewidth}
            \centering
            \includegraphics[width=1\textwidth]{SLIKE/figure9a}
            %\caption{Treniranje na većim bazama popravlja performanse}
   %         \caption{9-1-1-5 ($n_{22}=32$) i 9-1-1-1-5 ($+ n_{23}=16$)}
        \end{minipage}
        %\hspace{0.5cm}
        \begin{minipage}[b]{0.45\linewidth}
            \centering
            \includegraphics[width=1\textwidth]{SLIKE/figure9b}
            %\caption{Veći filteri 2. sloja su bolji}
        \end{minipage}
    \caption{Dublje strukture ne vode nužno boljim rezultatima}
    \end{figure}
\begin{figure}[ht]
        \begin{minipage}[b]{0.45\linewidth}
            \centering
            \includegraphics[width=1\textwidth]{SLIKE/figre10}
            %\caption{Treniranje na većim bazama popravlja performanse}
   %         \caption{9-1-1-5 ($n_{22}=32$) i 9-1-1-1-5 ($+ n_{23}=16$)}
        \end{minipage}
        %\hspace{0.5cm}
        \begin{minipage}[b]{0.45\linewidth}
            \centering
            \includegraphics[width=1\textwidth]{SLIKE/figure12}
            %\caption{Veći filteri 2. sloja su bolji}
        \end{minipage}
    \caption{Poređenja sa state-of-the-art tehnikama}
    \end{figure}

\end{frame}


\begin{frame}
\frametitle{Eksperimenti sa bojom}

\begin{table}[h!]
\centering
\begin{tabular}{ |c|c|c|c|c| } 
\hline
\multirow{2}{*}{Strategije}&\multicolumn{4}{|c|}{PSNR vrijednost kanala}\\
\cline{2-5}
&Y & Cb & Cr & RGB-slika\\ 
\hline
Bikubična s. & 30.39 & 45.44 & 45.42 & 34.57 \\
Y-samo s. & 32.39 & 45.44 & 45.42 & 36.37 \\
YCbCr & 29.25 & 43.30 & 43.39 & 33.47 \\
Y-pretreniranje & 32.19 & 46.49 & 46.45 & 36.32 \\
CbCr-pretreniranje & 32.14 & 46.38 & 45.84 & 36.25 \\
RGB & 32.33 & 46.18 & 46.20 & \textbf{36.44} \\
KK & 32.37 & 44.35 & 44.22 & 36.32 \\
\hline
\end{tabular}
\caption{Najbolje tretiranje svih kolor-kanala RGB strategije}
\label{table:2}
\end{table}

\end{frame}


\begin{frame}
\frametitle{Poboljšanje modela}

 \begin{figure}[h]
\includegraphics[width=\textwidth]{SLIKE/better}
\centering
\caption{Vizualizacija prelaza sa SRCNN na FSCRNN}
\label{fig:improv_1}
\end{figure}

\end{frame}
%\begin{frame}
%\frametitle{Outline}
%\tableofcontents
%\end{frame}

\begin{frame}
\frametitle{Zaključak}

\begin{itemize}
\item Jedan od ključnih standarda kvaliteta za SR problem
\item Zajednička optimizovana struktura koja obezbjeđuje funkciju mapiranja LR-HR
\item Jednostavna struktura i brzina izvršavanja
\item SRCNN i SC tehnike za SR problem su dosta slični
\item Istovremeno baratanje svim kolor-kanalima i pokazivanje superiornih rezultata
\item Pruža temelj novim i sve bržim tehnikama
\item Praktična i za druge probleme mašinskog vida, sa mnogim važnim primjenama u stvarnom životu (medicina)
\end{itemize} 

\end{frame}


\begin{frame}
\frametitle{Fin}

Hvala na pažnji!
 

\end{frame}


\end{document}